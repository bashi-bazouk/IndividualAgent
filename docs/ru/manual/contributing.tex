\chapter{Contributing}

You have gone through a complete description of all the features that Mongrel2 has right now,
but \emph{not} all the features it \emph{will} have.  I tend to write small software that does
exactly what it needs to do, and Mongrel2 is no different.  What you see here are the
majority of the things you need to do right now, and we'll be slowly adding things
people need.

Right now, what you have in your hands is something you can use, but it needs help.
There is not SSL support yet.  We're just using plain old poll to get things working.
There's no dynamic modules or filters yet (if we ever have them), and we haven't
really deployed it in anything important.

That means you can contribute.  If you want to help with Mongrel2, just go to
the \href{http://mongrel2.org}{Mongrel2} site and do whatever you can.  Submit bug
reports, help with docs, offer patches, join the mailing list, whatever.

The reason I am mostly begging people who are interested in Mongrel2 for help is that I'm
on a mission.  I've been through 3 or 4 major technology cycles that got us nowhere
but forced me and many other programmers to learn whole new languages with dubious
benefits.  I remember C++ wiping out C with no real benefit.  I remember Java wiping
out C++, again, with no real benefit.  I remember Ruby wiping out Java, and again, with no
real benefit.  In the end, the only thing we got from each ``language revolution'' was
yet another Ponzi Scheme we all had to pretend to believe in if we wanted to get a
job.

My goal with Mongrel2 is to lessen this suffering.  A professional engineer should not
have to follow fads like a fashion designer.  You should be able to use the tool that
works best for the environment, and all languages should be treated equally in the eyes
of core technologies like web servers and email servers.  Businesses should also be
able to hire programmers, not by what language they know, but by how good they are and
how well they fit.  Companies should be able to pivot off dead or poorly-suited
technologies and onto better ones without major upheaval.

I don't know if Mongrel2 will do this, but I do know it has a chance to, at least, start
the trend of language agnosticism in our industry.  If you want to help, then please,
come contribute.

Thanks,

\indent Zed
